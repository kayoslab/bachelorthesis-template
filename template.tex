%%%%%%%%%%%%%%%%%%% vorlage.tex %%%%%%%%%%%%%%%%%%%%%%%%%%%%%
%
% LaTeX-Vorlage zur Erstellung von Projekt-Dokumentationen
%
% Basis: Vorlage svmono des Springer Verlags
%
%%%%%%%%%%%%%%%%%%%%%%%%%%%%%%%%%%%%%%%%%%%%%%%%%%%%%%%%%%%%%

\documentclass[envcountsame,envcountchap, deutsch]{layout}

\usepackage{makeidx}         	% Index
\usepackage{multicol}        	% Zweispaltiger Index
\usepackage[bottom]{footmisc}	% Erzeugung von Fu�noten

\usepackage[pdftex]{graphicx}
\usepackage[pdftex,plainpages=false]{hyperref}

%%-----------------------------------------------------
\usepackage{color}				     % Farbverwaltung
%\usepackage{ngerman} 			% Neue deutsche Rechtsschreibung
\usepackage[english, ngerman]{babel}
\usepackage[latin1]{inputenc} 	% Erm�glicht Umlaute-Darstellung
%\usepackage[utf8]{inputenc}  	% Erm�glicht Umlaute-Darstellung unter Linux (je nach verwendetem Format)

%-----------------------------------------------------
\usepackage{listings} 			% Code-Darstellung
\lstset
{
	basicstyle=\scriptsize, 	% print whole listing small
	keywordstyle=\color{fhcolor2}\bfseries,
								% underlined bold black keywords
	identifierstyle=, 			% nothing happens
	commentstyle=\color{red}, 	% white comments
	stringstyle=\ttfamily, 		% typewriter type for strings
	showstringspaces=false, 	% no special string spaces
	framexleftmargin=7mm, 
	tabsize=3,
	showtabs=false,
	frame=single, 
	rulesepcolor=\color{blue},
	numbers=left,
	linewidth=146mm,
	xleftmargin=8mm
}
\usepackage{textcomp} 			% Celsius-Darstellung
\usepackage{amssymb,amsfonts,amstext,amsmath}	% Mathematische Symbole
\usepackage[german, ruled, vlined]{algorithm2e}
\usepackage[a4paper]{geometry} % Andere Formatierung
\usepackage{bibgerm}
\usepackage{array}

\usepackage{helvet}         % selects Helvetica as sans-serif font
\usepackage{courier}        % selects Courier as typewriter font
\usepackage{type1cm}        % activate if the above 3 fonts are  not available on your system
\hyphenation{Ele-men-tar-ob-jek-te  ab-ge-tas-tet Aus-wer-tung House-holder-Matrix Le-ast-Squa-res-Al-go-ri-th-men} 
% Weitere Silbentrennung bei Bedarf angeben
\setlength{\textheight}{1.1\textheight}
\pagestyle{myheadings} 			% Erzeugt selbstdefinierte Kopfzeile
\makeindex 						% Index-Erstellung


%--------------------------------------------------------------------------
\begin{document}
%------------------------- Titelblatt -------------------------------------
\title{Lorem ipsum dolor sit amet}
\subtitle{consetetur sadipscing elitr sed diam}
\author{Simon Christian Kr\"{u}ger}  % Autor der Arbeit
%--------------------------------------------------------------------------

\major{Bachelor of Science (B.Sc)}  % Studiengang inkl. Abk�rzung
\subject{Wirtschaftsinformatik} % Studienfach
\faculty{Wirtschaft und Gesundheit} % Fachbereich

\period{Sommersemester 2016}
\firstexaminer{Titel Vorname Name} 		  % Erstpr�fer/in
\secondexaminer{Titel Vorname Name}   % Zweitpr�fer/in
\address{G\"{u}tersloh} 			 % Datum
\submitdate{01.01.1970} 				% Abgabedatum

\begingroup
  \renewcommand{\thepage}{Titel}
  \mytitlepage
  \newpage
\endgroup

\cleardoublepage % if some contents comes before
%\pagenumbering{Roman}


%--------------------------------------------------------------------------
\frontmatter 
%--------------------------------------------------------------------------


\chapter{Sperrklausel}
%% deutsch
\paragraph*{}
Die Sperrklausel wird nach der Vorlage des Fachbereichs zeitnah nachgereicht.

%% englisch
\paragraph*{}
The same in english.
\input{chapters/Vorwort}				% Vorwort (optional)
\input{chapters/Kurzfassung} 			% Kurzfassung Deutsch/English

%--------------------------------------------------------------------------
\tableofcontents 						% Inhaltsverzeichnis
\addcontentsline{toc}{chapter}{\contentsname}

\listoffigures 							% Abbildungsverzeichnis (optional)
\addcontentsline{toc}{chapter}{\listfigurename}

\listoftables 							% Tabellenverzeichnis (optional)
\addcontentsline{toc}{chapter}{\listtablename}


%--------------------------------------------------------------------------
\mainmatter                        		% Hauptteil (ab hier arab. Seitenzahlen)
%--------------------------------------------------------------------------
% Die Kapitel werden in separaten .tex-Dateien abgelegt und hier eingebunden.
\input{chapters/EinleitungProblemstellung}
\input{chapters/WeitereKapitel}
\input{chapters/Bausteine}
\input{chapters/Beispiel}
\input{chapters/ZusammenfassungAusblick}

%--------------------------------------------------------------------------

\begin{appendix}
	%--------------------------------------------------------------------------
   \backmatter                        		% Anhang
   %-------------------------------------------------------------------------
   \bibliographystyle{geralpha}			% Literaturverzeichnis
   \bibliography{literature}     			% BibTeX-File literature.bib
   
   \printindex 							% Index (optional)
   
   \include{chapters/Glossar} % Glossar   
   \chapter{Versicherung der Kandidatin / des Kandidaten}

Ich versichere, dass ich die vorstehende Arbeit selbstst�ndig verfasst und mich fremder Hilfe nicht bedient habe. Alle Stellen, die w\"{o}rtlich oder sinngem\"{a}{\ss} ver�ffentlichtem oder nicht ver\"{o}ffentlichtem Schrifttum entnommen sind, habe ich als solche kenntlich gemacht.\\

\vspace{2cm}

\begin{minipage}[t]{3cm}
\rule{3cm}{0.5pt}
Datum
\end{minipage}
\hfill
\begin{minipage}[t]{9cm}
\rule{9cm}{0.5pt}
Unterschrift der Kandidatin / des Kandidaten
\end{minipage}	% Selbstst�ndigkeitserkl�rung

\end{appendix}

\end{document}
